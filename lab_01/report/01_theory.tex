\chapter*{Цель работы}
Целью работы является получение навыков решения задачи Коши для ОДУ методами Пикара и явными методами первого порядка точности (Эйлера) и второго порядка точности (Рунге-Кутта).

\newpage
\chapter{Теоретические сведения}

Пусть дано ОДУ (Обыкновенное Дифференциальное уравнение) n-ого порядка по формуле (\ref{eq:ody}).

\begin{equation}
	F(x, u', u'', ... , u^{(n)} = 0)
	\label{eq:ody}
\end{equation}

ОДУ любого порядка может быть сведено к системе ОДУ 1-ого порядка.

\section{Задача Коши}

\textit{Задача Коши} состоит в нахождении решения дифференциального уравнения, удовлетворяющего начальным условиям. 
Это одна из основных задач теории дифференциальных уравнений.

Имеется задача Коши по формуле (\ref{eq:ref1}).

\begin{equation}
	{\begin{cases}
			u'(x) = f(x,u) \\
			u(\xi) = \eta
	\end{cases}}
	\label{eq:ref1}
\end{equation}

Методы решения ОДУ в задачи Коши:

\begin{enumerate}
	\item аналитические;
	\item приближенно-аналитические;
	\item численные.
\end{enumerate}

\newpage
\section{Методы решения задачи Коши}

При отсутствии аналитического решения можно воспользоваться приближенно-аналитическим методом Пикара.
Заменив дифференциальное уравнение интегральным получим (\ref{eq:ref2}).

\begin{equation}
	y(x)^{(s)} = \eta + \int_{\xi}^{x} f(t, y^{s-1}(t)) dt
	\label{eq:ref2}
\end{equation}

\begin{equation}
	y^{(0)} = \eta
\end{equation}

Метод сходится если:

\begin{enumerate}
	\item правая часть непрерывная;
	\item выполнено условие Липшица (\ref{eq:Lip})
\end{enumerate}

\begin{equation}
	|f(x, u_1) - f(x, u_2| \leq L |u_1-u_2|
	\label{eq:Lip}
\end{equation}
где L - константа Липшица.

Метод Эйлера (\ref{eq:ref3}).

\begin{equation}
	y_{n+1} = y_n + h\cdot f(x_n, y_n)
	\label{eq:ref3}
\end{equation}


Метод Рунге-Кутты (\ref{eq:ref4}).

\begin{equation}
	y_{n+1} = y_n + h\cdot[(1-\alpha)k_1 + \alpha \cdot k_2]
	\label{eq:ref4}
\end{equation}

Где $k_1$ и $k_2$ представлены как (\ref{eq:ref5}) и (\ref{eq:ref6}) соответственно. А $\alpha$ = 1 или $\frac{1}{2}$

\begin{equation}
	k_1 = f(x_n, y_n)
	\label{eq:ref5}
\end{equation}

\begin{equation}
	k_2 = f(x_n + \frac{h}{2\alpha}, y_n + \frac{h}{2\alpha}k_1)
	\label{eq:ref6}
\end{equation}

\chapter{Задание и вычисления приближений для метода Пикара}

Дана формула {\ref{eq:task}}.

\begin{equation}
	{\begin{cases}
			u'(x) = u^2 + x^2 \\
			u(0) = 0
	\end{cases}}
	\label{eq:task}
\end{equation}

Используя описанные выше методы построить таблицу для:

\begin{enumerate}
	\item Метода Пикара I-IV приближений.
	\item Метод Эйлера.
	\item Метод Рунге-Кутты.
\end{enumerate}

Приближения представлены в формулах (\ref{eq:picar_1}), (\ref{eq:picar_2}), (\ref{eq:picar_3}) и (\ref{eq:picar_4}):

\begin{equation}
	\label{eq:picar_1}
	y^{(1)} = 0 + \int_{0}^{x} t^2 dt= \frac{x^3}{3} 
\end{equation}

\begin{equation}
	\label{eq:picar_2}
	y^{(2)} = 0 + \int_{0}^{x}\left[ \left( \frac{t^3}{3} \right)^2 + t^2 \right] dt = \frac{x^3}{3} + \frac{x^7}{63}
\end{equation}

\begin{equation}
	\label{eq:picar_3}
	\begin{split}
		y^{(3)} = 0 + \int_{0}^{x}\left[ \left(\frac{t^7}{63} + \frac{t^3}{3} \right)^2 + t^2 \right] dt = 
		\int_{0}^{x}\left[ \frac{t^{14}}{63^2} + \frac{2}{63*3}t^{10} + \frac{t^6}{9} + t^2 \right] = \\
		\frac{x^3}{3} + \frac{x^7}{63} + \frac{2x^{11}}{2079} + \frac{x^{15}}{59535}
	\end{split}
\end{equation}

\begin{equation}
	\label{eq:picar_4}
	\begin{split}
		y^{(4)} = 0 + \int_{0}^{x}\left[ \left( \frac{t^3}{3} + \frac{t^7}{63} + \frac{2t^{11}}{2079} + \frac{t^{15}}{59535} \right)^2 + t^2 \right] dt = \\ 
		\frac{x^3}{3} + \frac{x^{7}}{63} + \frac{2x^{11}}{2079} + \frac{13x^{15}}{218295} + \frac{82x^{19}}{37328445} + 
		+ \frac{662x^{23}}{10438212015} + \\
		+ \frac{4x^{27}}{3341878155} + \frac{x^{31}}{109876903905}
	\end{split}
\end{equation}